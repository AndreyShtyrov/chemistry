\documentclass[12pt]{article}
\usepackage[T2A]{fontenc}
\usepackage[utf8]{inputenc}
\usepackage[russian]{babel}
\usepackage{algorithm}
\usepackage{algpseudocode}
\usepackage{amssymb}
\usepackage{amsmath}
\usepackage{breqn}
\usepackage{enumitem}

\usepackage{graphicx}
\graphicspath{ {./} }

\def\multiset#1#2{\ensuremath{\left(\kern-.3em\left(\genfrac{}{}{0pt}{}{#1}{#2}\right)\kern-.3em\right)}}
\newcommand{\RNum}[1]{\uppercase\expandafter{\romannumeral #1\relax}}

\title{Алгоритмы. Домашнее задание 1}
\author{Новиков Георгий \\ Группа 505, биоинформатика}

\begin{document}

\maketitle
Меня интересует следующий вопрос. Дана функция
$$f : R^n \rightarrow R$$
Про функцию известно, что существует $k$ семейств кривых, на которых функция остается постоянной, заданных, например, параметризацией:
$$g_i : R^n \times R \rightarrow R^n, i = 1, \dots, k$$
$$\begin{cases} f(x) = f(g_i(x, \alpha)) \forall x, \alpha \\ g_i(x, 0) = x\end{cases}$$
Я хочу найти такую замену координат $x(t_1, \dots, t_{n - k})$, которая бы параметризовала кусок пространства, избавив функцию $f$ от лишнихстепеней свободы, при этом не потеряв информации.
(Я понимаю, что последнее требование я формулирую не строго, и даже н математически, но я плохо понимаю, как я должен его сформулировать). \\
Например, пусть функция зависит от трех переменных $f(x, y, z)$, и известно, что:
$$\forall \alpha: f(x, y, z) = f(x, y, z + \alpha)$$
$$\forall \alpha: f(x, y, z) = f(x \cos(\alpha) - y \sin(\alpha), x \sin(\alpha) + y \cos(\alpha), z)$$
Тогда фактически, функция не зависит от $z$, а в оставшихся $x$ и $y$ важен лишь радиус, и тогда искомая мной замена координат может выглядит как, например,
$$x(t) = t \\ y(t) = z(t) = 0$$
Что переведет нам функцию $f(x, y, z)$ в гораздо более удобный вид $f(t) = f(t, 0, 0)$ без потери информации (под информацией я здесь понимаю разного рода особенности $f$, например минимумы, максимумы и т.д.). В таком виде явно должно быть проще численно искать экстремумы функции, гессиан не будет вырождаться, меоды будут быстрее сходиться и т.д.. \\
Вот такую задачу я бы хотел уметь решать в общем случае. Это мой первый вопрос. Мой второй вопрос, имеющий непосредственное отношение к первому. Мое решение: попробуем найти такую замену координат, которая в каждой точке будет ортогональна эквипотенциальным кривым данной точки. В моих обознрачениях это должно означать, что
$$x(t_1, \dots, t_{n - k})$$
$$\forall i \in \{1, \dots, k\}, j \in \{1, \dots, n - k\}: \frac{\partial{x^T(t_1, \dots, t_{n - k})}}{\partial{t_j}} \cdot \frac{\partial{g_i(x, \alpha)}}{\partial{\alpha}} = 0$$
То есть скалярное произведение направления по координате $t_j$ и касательной к кривой $g_i(x(t_1, \dots, t_j), \alpha)$ равно нулю. Фактически, $x(t_1, \dots, t_{n - k})$, удовлетворяющее верхнему условию является параметризацией некоторого подпространство, потому что допускает любую замену координат $t(\theta_1, \dots, \theta_{n - k})$, ведь выражение $\frac{\partial{x^T}}{\partial \theta_j}$ распишется как $\sum_k \frac{\partial{x^T}}{\partial t_k} \frac{\partial t_k}{\partial \theta_j}$ (значит равенство скалярного произведения нулю остается верным). \\
Это приводит нас к системе дифференциальных уравнений, у которой мы хотим найти какое-нибудь нетривиальное решение. 

\end{document}
